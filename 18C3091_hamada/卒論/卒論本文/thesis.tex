\documentclass[a4paper,12pt,report]{jsbook}
%\usepackage[top=30truemm,bottom=30truemm,left=25truemm,right=25truemm]{geometry}

\usepackage{graphicx}
\usepackage{amsmath}
\usepackage{amsfonts}
\usepackage{amssymb}
\usepackage{algorithm}
\usepackage{algorithmic}
\usepackage{url}
\usepackage{bm}
\usepackage{color}
\newcommand{\argmax}{\mathop{\rm arg~max}\limits}
\newcommand{\argmin}{\mathop{\rm arg~min}\limits}

%ページに綴じ代をつくる
\setlength{\textwidth}{38zw}
\setlength{\oddsidemargin}{0.5cm}


%共通設定
\renewcommand{\bibname}{参考文献}
\renewcommand{\theenumi}{\roman{enumi}}
\usepackage{makeidx}
\pagestyle{headings}
\usepackage{graphicx}
\pagenumbering{roman}

%枠関連の設定
\usepackage{fancybox}
\usepackage{framed}
\newenvironment{longbox}{%
  \def\FrameCommand{\fboxsep=\FrameSep \fbox}%
  \MakeFramed {\FrameRestore}}%
 {\endMakeFramed}

\makeindex

\begin{document}


%
% 表紙
%
\begin{titlepage}
\begin{center}
{\large 令和04度卒業論文\\}
\vspace{2.5cm}
{\fontsize{36pt}{40pt}\selectfont%
 T-SVDによる\\テンソルノイズ除去における直交変換の\\SUREによる最適化\\}
\vspace{3cm}
{\large%
  千葉工業大学 先進工学部 \\
  \vspace{3mm}18C03091 濱田 崚平\\
  \vspace{2cm}指導教員 宮田 高道 教授\\
  \vspace{5mm}提出日 令和4年1月17日\\
}%各所にあるvspaceの値を調整して1ページに収まるようにしましょう.
\end{center}
\end{titlepage}

%%%%%%%%%%%%%%%%%%%%%%%%%%和文要旨%%%%%%%%%%%%%%%%%%%%%%%%%%%%%%
\begin{titlepage}
\begin{longbox}
\begin{center}
 {\Huge T-SVDによるテンソルノイズ除去における直交変換のSUREによる最適化}\\
 {\large%
   \vspace{3mm}%
   18C3091 濱田 崚平\\
 }
\end{center}
{\bf 概要:}%
T-SVDを用いたテンソル復元は,テンソルに離散フーリエ変換を行なったのち,行列に分解してそれぞれの行列を低ランク行列で近似することでテンソルを復元する手法である.これに対して,従来用いられている離散フーリエ変換を,その他の直交変換(離散コサイン変換や主成分分析の変換行列など)で置き換えることが可能であることが知られており,それによって高精度なテンソル復元を行える手法も提案されている.その一方で,従来の研究により,復元対象のテンソルによって最適な直交変換が異なることも明らかとなっている.そこで本研究では,テンソル復元の一種であるテンソルノイズ除去を対象として,ノイズ除去後の最小二乗誤差を(復元対象の原テンソルを参照することなく)推定することを可能とするSUREを用いて対象のテンソルごとに候補中から最適な直交変換を選択する手法を提案する.これにより,従来手法と比較して,常に高い精度でテンソルノイズ除去が行うことが可能となった.
\\
\\
\end{longbox}
\end{titlepage}
%%%%%%%%%%%%%%%%%%%%%%%%%%%%%%%%%%%%%%%%%%%%%%%%%%%%%%%%%%%%%%%%

%%%%%%%%%%%%%%%%%%%%%%%%%%%%%%%%%%%%%%%%%%%%%%%%%%%%%%%%%%%%%%%%


\bibliographystyle{junsrt}
%\maketitle
\tableofcontents


\newpage
\pagenumbering{arabic}
\setcounter{page}{1}

%%%%%%%%%%%%%%%%%%%%%%%ここから本文を書きましょう%%%%%%%%%%%%%%%
%各章毎にファイルを分割して\inputで読み込ませると管理が楽
\chapter{序論}
\section{背景}
テンソルノイズ除去とは,低ランクであると仮定できる原テンソルにノイズテンソルを加算した観測テンソルを入力として,原テンソルを推定する手法である.テンソルノイズ除去を実現する既存手法のひとつとして,t-SVD\cite{Kilmer}を用いる手法が挙げられる.この手法が用いるt-SVDとは,主に3階テンソルを入力とし,入力テンソルの第三次元に対して離散フーリエ変換(DFT)を行なったのち,テンソルを行列へと分解し,それらの行列を特異値分解することでテンソルの特異値分解を実現するものである.このとき,得られた特異値に対してしきい値処理を行うことでテンソルデノイズが実現できることが知られており,デノイズだけでなく一般のテンソル復元にも応用することができる\cite{Lu}.
この手法をさらに発展させ,従来用いられているDFTを,離散コサイン変換(DCT)および主成分分析(PCA)に置き換えることでテンソル復元性能を高めることが提案されている\cite{DCT, kohama}.一方で,一般のテンソル復元の一種である低ランクテンソル復元を対象とした先行研究\cite{kohama}では,復元対象のテンソルによって最適な直交変換が異なることが明らかとなっている.
\section{目的}
上記の背景より,本研究では復元対象のテンソルに適したノイズ除去結果を得ることを目的とし,DFT, DCT,PCAのなかから最良の直交変換を選択することで,これを実現する手法を提案する.提案手法では,前述の3つの直交変換のうち,最小二乗誤差の意味で原テンソルに最も近いテンソルを得られるものを最良の変換と定義する.しかしながら,実際にテンソルノイズ除去を行う際には原テンソルを参照することは不可能である.そこで本研究では,推定した行列と真の行列の平均二乗誤差を不偏推定するStein's unbiased risk estimator (SURE)を用いることでこの問題を解決する.
\chapter{関連研究}
\section{t-SVD}
テンソル$\mathcal{A}\in \mathbb{R}^{n_1\times n_2\times n_3}$のt-SVD\cite{Kilmer}は,以下のように定義される.
\begin{equation}
    \label{tsvd}
    \mathcal{A} = \mathcal{U} * \mathcal{S} * \mathcal{V}^*
\end{equation}

ここで $*$はt-積であり,
$\mathcal{A}*\mathcal{B} = \mathrm{fold}(\mathrm{bcirc}(A)・\mathrm{unfold}(B)$を表している.また,
$\mathcal{V}^*$は$\mathcal{V}$の転置テンソルであり,$\mathcal{V}\in \mathbb{R}^{n_1\times n_2\times n_3}
$
ならば,$\mathcal{V}^*\in \mathbb{R}^{n_2\times n_1\times n_3}$である.
ここで,
\begin{equation}
    \label{eq:yx}
    \mathrm{bcirc}(\mathcal{A})=
    \begin{bmatrix}
A^{(1)} &  A^{(n_3)} & \cdots & A^{(2)}\\
A^{(2)} &  A^{(1)}   &        & A^{(3)} \\
\vdots  &            & \ddots & \vdots \\
A^{(n_3)}  &  A^{(n_3-1)}     & \cdots & A^{(1)}
\end{bmatrix}
\end{equation}
\\

\begin{equation}
    \label{unfold}
    \mathrm{unfold}(\mathcal{A}) = 
    \begin{bmatrix}
A^{(1)} \\
A^{(2)} \\
\vdots  \\
A^{(n_3)}
\end{bmatrix},\mathrm{fold}(\mathrm{unfold}(\mathcal{A})=\mathcal{A}
\end{equation}である.
また,$A^{(i)}$はテンソル$\mathcal{A}$のフロンタルスライス(行列)をさしており,
$A^{(i)} = \mathcal{A}(:,:,i)$
と定義される.すなわち,式(\ref{unfold})で定義されるunfoldは,テンソルをフロンタルスライスに分解し,それらを縦に連結した行列を得る処理である.
いま,$\bar{\mathcal{A}}$をテンソル$\mathcal{A}$の第3次元方向に対してDFTを施して得られるテンソル\footnote{Matlabの記法を使えば$\bar{\mathcal{A}}=\mathrm{fft}(\mathcal{A}} [],3)$とかける.}であるとすると,$\mathcal{C}=\mathcal{A}*\mathcal{B}$であれば,任意の$i$について
\begin{equation}
\label{dft}
 \bar{C}^{(i)} =  \bar{A}^{(i)}   \bar{B}^{(i)} 
\end{equation}であることは明らかである.これにより,テンソル$\mathcal{A}$のt-SVDは,分解対象のテンソル$\mathcal{A}$の第3次元(フロンタル方向)に対してDFTを行ったのち,各フロンタルスライスを行列とみなして(行列の)特異値分解を行うことで実現できる.すなわち
\begin{equation}
\mathcal{A}=\mathcal{U}*\mathcal{S}*\mathcal{V}^*
\end{equation}
であれば,任意のiについて

\begin{equation}
\label{frontalslice_use_t-svd}
\bar{A}^{(i)} = \bar{U}^{(i)} \bar{S}^{(i)} \bar{V}^*^{(i)}
\end{equation}が成立する.



\section{PCAを用いたt-SVD}
上記のt-SVDにおいて,テンソル第3次元方向に対して適用していたDFTを,DCTやPCAで置き換える手法が提案されている.本節では,
PCAを使った手法について解説する.いま,対象の3階テンソルを$\mathcal{A}\in \mathbb{R}^{n_1\times n_2\times n_3}$とする.このテンソルの(第三次元方向に対する)PCAは

\begin{equation}
    \label{eq:yx}
    \bar{\mathcal{A}}(i,j,:)=V\mathcal{A}(i,j,:).
\end{equation}によって実現できる.ここで,$\bar{\mathcal{A}}$はPCA後のテンソルであり,$\mathcal{A}(i,j,:)$はテンソル$\mathcal{A}$のフロンタルファイバーを列ベクトルとみなしたものであるとする.また,$n_3\times n_3$の行列$V$は以下のように定義される共分散行列 $C \in \mathbb{R}^{n_3 \times n_3}$の単位固有ベクトルからなる変換行列をさしている.

\begin{equation}
    \label{eq:yx}
    C_{i,j} = \frac{1}{n_1n_2} \mathrm{Tr}(A^{(i)}-\bm{\mu})^T(A^{(j)}-\bm{\mu}))
\end{equation}ここで,ベクトル$\bm{\mu}_k$は$k$番目のフロンタルスライスの平均値であり,

\begin{equation}
    \label{eq:yx}
    \mu_k = \frac{1}{n_1n_2} \sum_{i=1}^{n_1}\sum_{j=1}^{n_2}A_{ijk}
\end{equation}と定義される.



\section{T-SVDによるテンソルデノイズ}
いま,未知の真のテンソル$\mathcal{X} \in \mathbb{R}^{n_1 \times n_2 \times n_3}$に対して平均0,分散$\sigma^2$ の正規分布に従うノイズテンソル$\mathcal{W}$が付加された観測テンソル$\mathcal{Y}=\mathcal{X}+\mathcal{W}$が得られているとする.T-SVDを用いたテンソルデノイズは観測テンソルから真のテンソル$\mathcal{X}$を推定手法の一種であり,$\mathcal{X}$の推定値$\^{\mathcal{X}}$は式(\ref{xhat})に示す最小化問題を解くことによって得られる.

\begin{equation}
    \label{xhat}
    \hat{\mathcal{X}} = \mathcal{\arg \min_Y||Y||_*^Q}
\end{equation}ここで, $||\cdot||_*^Q$は(直交変換としてQを用いたときの)テンソル核ノルムであり,以下のように定義される.

\begin{equation}
    \label{tnn}
    ||\mathcal{X}||_*^Q =  \sum_{i=1}^{n_3}||\bar{X}^{(i)}||_* , (i=1,...,n_3)
\end{equation}

ここで,$X^{(i)}$はの第3軸方向に対して直交変換Qを施したものの$i$番目のフロンタルスライスである.また,$||\cdot ||_*$は行列の核ノルムであり,特異値の和を表している.

このとき,式(\ref{xhat})の解$\^{\mathcal{X}}$の直交変換後の各フロンタルスライスは以下の式,

\begin{equation}
    \label{eq:yx}
    \hat{\bar{X}}^{(i)} = \mathrm{SVT}_\lambda (\bar{X}^{(i)})= \bar{U}^{(i)} T_\lambda(\bar{S}^{(i)}) (\bar{V}^{(i)})^*
\end{equation}で得られる.ここで,しきい値処理$T_\lambda$は以下のように定義される.
\begin{equation}
    \label{eq:yx}
    T_\lambda(S)(i,i) = \max (S(i,i) -\lambda,0)
\end{equation}
このような特異値に対するしきい値処理はSVT(singular value thresholding)とよばれる.




\section{核ノルム最小化におけるSURE}
SURE(Stein's unbiased risk estimator)は原画像を用いることなく,推定画像との平均二乗誤差を不偏推定することができる手法である.

いま,原行列$X \in \mathbb{R}^{m \times n$にノイズ行列$W\in \mathbb{R}^{m \times n$が加算された観測行列$Y=X+W$から$X$を前述のSVT用いて推定するとものする.このようなSVTを用いた行列デノイズにおけるSUREはCandesらの検討によって以下のように明らかとなっている\cite{Candes}.
\begin{equation}
\begin{split}
    \label{sure}
    \mathrm{SURE}(\mathrm{SVT}_\lambda(Y)) = -mnr^2 + {\sum_{i=1}^{\mathrm{min}(m,n)}} \mathrm{min}(\lambda^2,\sigma_i^2) 
    \\+ 2r^2 \mathrm{div}(\mathrm{SVT}_\lambda(Y))
\end{split}
\end{equation}ここで,$r^2$はガウシアンノイズの分散,
$\lambda$は正のスカラー,$\{\sigma_i\}^n_{i=1}$は$Y$の特異値を表す.さらに,$\mathrm{div}(\mathrm{SVT}_\lambda(Y))$は$(\mathrm{SVT}_\lambda(Y))$の自由度と呼ばれる値であり,これは以下のように得られる.

\begin{equation}
\begin{split}
    \label{eq:yx}
    \mathrm{div}(\mathrm{SVT}_\lambda(Y)) = |m-n| {\sum_{i=1}^{\mathrm{min}(m,n)}}  (\sigma_i> \lambda)_+ 
    \\ +  2  \sum_{i≠j,i,j=1}^{\mathrm{min}(m,n)} \frac{\sigma_i(\sigma_i-\lambda)_+}{\sigma_i^2-\sigma_j^2}
    \\
\end{split}
\end{equation}

本研究では,この行列デノイズのSUREを利用して,t-SVDを用いたテンソルデノイズのSUREを実現する.
\chapter{提案手法}
\section{テンソルノイズ除去におけるSURE}
ここでは2.4で紹介したSVTのSUREをテンソルに対して行う方法を説明する.
2.3節と同様に,原テンソルを$\mathcal{X}$とするとし,観測テンソル$\mathcal{Y}$が
$\mathcal{Y}=\mathcal{X}+\mathcal{W}$
からSVTによって復元テンソルが得られているものとする.このとき,復元した$\hat{\mathcal{{X}}}$と原テンソルのあいだのSUREは,
\begin{equation}
\label{mysure}
\sum^{n_3}_{i=1} \frac{\mathrm{SURE}(\mathrm{SVT}_\lambda (\bar{X}^{(i)}))}{n_3}
\end{equation}である.これは,直交変換の前後ではベクトル間の距離が保たれること(パーセバルの関係)ならびにフロンタルスライスがテンソルを互いに素な要素に分割していることによる.

\section{SUREを用いた最適な直交変換の選択法}

これまでの内容を踏まえ,以下のように最適な直交変換を選択する手法を提案する.
\vskip\baselineskip
\vskip\baselineskip
ステップ1:式(\ref{dft})で行うDFTをPCAやDCTに置き換え,それらの三種類の直交変換を用いて式(\ref{frontalslice_use_t-svd})のように特異値分解を行い,式(\ref{xhat})で紹介した最小化問題をそれぞれ解くことで,観測テンソル$\mathcal{Y}$のデノイズを行う.
\vskip\baselineskip

ステップ2:ノイズ除去後の各テンソルに対し,式(3.1)を用いてSUREを計算する.パラメータ$\lambda$については,直交変換ごとにSUREが最小となる値(以下,最適$\lambda$)を選択するものとする.

\vskip\baselineskip
ステップ3:各直交変換において最適$\lambda$を用いたときのSUREを比較し,最も小さい値となったものを最適な直交変換とする.


\chapter{実験}
\section{実験概要}
本実験に用いる原テンソル$X∈\mathbb{R}^{n_1\times n_2\times 3}$ は
サイズ256×256×3の標準カラー画像12枚を用いた.このようなカラー画像では,第3次元(RGBカラーチャンネル)の間に強い相関があることが知られている.また,
ノイズテンソルとしては,各成分が独立同一の分散$\sigma=0.2, 0.5$の正規分布に従うものを用い,これを原テンソルに付加したものを観測テンソルとする.
比較手法としては,直交変換として常にDFT,DCT,PCAを用いる手法を選び,その性能を,SURE用いて最良の直交変換を選択する提案手法と比較する.復元性能の評価は原テンソルとノイズ除去後のテンソルとの間のPSNR(単位は[dB])によって行う.

\section{実験結果}
実験により得られた結果を表4.1,および表4.2に示す.これらの表から,SUREを用いて決定された直交変換は画像12枚のうち,11枚において最良の直交変換が選ばれることが明らかとなった.また,提案手法が最良の直交変換を選べていない場合においても,2番目にPSNRが大きい直交変換との差は0.05以下とごく小さいことが明らかとなった.
以上より,SUREを用いた直交変換の選択手法の有効性が明らかとなった.

\begin{table}[h]
\caption{標準画像に対するテンソルデノイズの性能比較($\sigma=0.2$,単位はPSNR,赤字は一番目に高い値,青字は二番目に高い値)}
\label{tab:settings}
\centering
\begin{tabular}{c||c|c|c||c}\hline
- & DFT & DCT & PCA & SURE\\\hline
Aerial & 21.20 & \textcolor{blue}{21.31} & \textcolor{red}{21.33} & \textcolor{red}{21.33}\\\hline
Airplane & 21.81 & \textcolor{blue}{21.86} & \textcolor{red}{22.14}& \textcolor{red}{22.14}\\\hline
Balloon & 24.11 & \textcolor{red}{24.19} & \textcolor{blue}{24.15} & \textcolor{blue}{24.15} \\\hline
Earth & 22.51 & \textcolor{blue}{22.59} & \textcolor{red}{22.87} & \textcolor{red}{22.87}  \\\hline
Girl & 23.09 & \textcolor{blue}{23.16} & \textcolor{red}{23.19} & \textcolor{red}{23.19}  \\\hline
Lenna & 21.69 & \textcolor{blue}{21.73} & \textcolor{red}{21.96} & \textcolor{red}{21.96} \\\hline
Mandrill &20.25 & \textcolor{red}{20.39} & \textcolor{blue}{20.34} & \textcolor{red}{20.39}  \\\hline
Parrots & 21.83 & \textcolor{blue}{21.90} & \textcolor{red}{21.92}  & \textcolor{red}{21.92}  \\\hline
Pepper & 20.70 & \textcolor{blue}{20.72} & \textcolor{red}{20.97} & \textcolor{red}{20.97}  \\\hline
Sailboat & 21.50 & \textcolor{blue}{21.57} & \textcolor{red}{21.97} & \textcolor{red}{21.97}  \\\hline
couple & 23.94 & \textcolor{blue}{24.05} & \textcolor{red}{24.11} & \textcolor{red}{24.11}  \\\hline
milkdrop &23.14 & \textcolor{blue}{23.19} & \textcolor{red}{23.48} & \textcolor{red}{23.48} \\\hline


\end{tabular}
\end{table}

\begin{table}[h]
\caption{標準画像に対するテンソルデノイズの性能比較($\sigma=0.5$,単位はPSNR,赤字は一番目に高い値,青字は二番目に高い値)}
\label{tab:settings}
\centering
\begin{tabular}{c||c|c|c||c}\hline
- & DFT & DCT & PCA & SURE\\\hline
Aerial &\textcolor{blue}{18.40}&	\textcolor{red}{18.46}&	18.21&	\textcolor{red}{18.46}\\\hline
Airplane &\textcolor{blue}{18.07}&\textcolor{red}{18.09}&	17.90&	\textcolor{blue}{18.07}\\\hline
Balloon &\textcolor{blue}{20.64}	&\textcolor{red}{20.70}&	20.28	&\textcolor{red}{20.70}\\\hline
Earth &18.65&	\textcolor{blue}{18.86}&	\textcolor{red}{18.94}&	\textcolor{red}{18.94}\\\hline
Gial&19.12&	\textcolor{blue}{19.22}&	\textcolor{red}{19.22}&	\textcolor{red}{19.22}\\\hline
Lenna &17.50&	\textcolor{blue}{17.55}&	\textcolor{red}{17.67}&	\textcolor{red}{17.67}\\\hline
Mandrill& 16.91&\textcolor{red}{17.06}&	\textcolor{blue}{16.97}	&\textcolor{red}{17.06}\\\hline
Parrots &17.52&	\textcolor{blue}{17.61}&	\textcolor{red}{17.67}	&\textcolor{red}{17.67}\\\hline
Pepper &16.24&	\textcolor{blue}{16.32}&	\textcolor{red}{16.66}&\textcolor{red}{16.66}\\\hline
Sailboat& 17.43	&\textcolor{blue}{17.53}	&\textcolor{red}{17.68}&	\textcolor{red}{17.68}\\\hline
couple &20.60&	\textcolor{blue}{20.67}&	\textcolor{red}{20.70}&	\textcolor{red}{20.70}\\\hline
milkdrop &17.32&	\textcolor{blue}{17.34}&	\textcolor{red}{17.48}	&\textcolor{red}{17.48}\\\hline
\end{tabular}
\end{table}
\chapter{結論}
本研究では,観測テンソルに応じたノイズ除去処理が行える手法を得ることを目的とし,SUREを用いることでt-SVDにおけるDFT,DCT,PCA
の三つの直交変換の中から最もノイズ除去性能が高い直交変換を選択する手法を提案した.実験では,12枚のカラー画像にそれぞれ三つの直交変換を固定的に用いてテンソルノイズ除去を行う手法と提案手法とを比較した.その結果,提案手法はほぼすべての画像において最良の直交変換を選択できることが明らかとなり,そうでない場合においても,2番目にPSNR が大きい直交変換との差はごく小さいことがわかった.

この結果から,テンソルノイズ除去において,SUREを用いて最良の直交変換を選択することが有効であることが明らかとなり,目的である観測テンソルに適したノイズ除去処理が行えることが確認された.提案手法の有効性を示した.


%%%%%%%%%%%%%%%%%%%%%%%%%%%%%%%%%%%%%%%%%%%%%%%%%%%%%%%%%%%%%%%%

\chapter*{謝辞}
\addcontentsline{toc}{chapter}{謝辞}
本研究を進めるにあたり,たくさんの助言やご指導をして下さいました宮田高道教授に深く感謝いたします,また,研究室の先輩方もたくさんの助言やご指導をしてくださり,感謝いたします.


\nocite{*}
\cleardoublepage
\pagestyle{plain}

\bibliography{ref}


%図の一覧を出力する場合
%\listoffigures

%表の一覧を出力する場合
%\listoftables

\end{document}

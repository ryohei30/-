\chapter{結論}
本研究では,観測テンソルに応じたノイズ除去処理が行える手法を得ることを目的とし,SUREを用いることでt-SVDにおけるDFT,DCT,PCA
の三つの直交変換の中から最もノイズ除去性能が高い直交変換を選択する手法を提案した.実験では,12枚のカラー画像にそれぞれ三つの直交変換を固定的に用いてテンソルノイズ除去を行う手法と提案手法とを比較した.その結果,提案手法はほぼすべての画像において最良の直交変換を選択できることが明らかとなり,そうでない場合においても,2番目にPSNR が大きい直交変換との差はごく小さいことがわかった.

この結果から,テンソルノイズ除去において,SUREを用いて最良の直交変換を選択することが有効であることが明らかとなり,目的である観測テンソルに適したノイズ除去処理が行えることが確認された.提案手法の有効性を示した.
\chapter{序論}
\section{背景}
テンソルノイズ除去とは,低ランクであると仮定できる原テンソルにノイズテンソルを加算した観測テンソルを入力として,原テンソルを推定する手法である.テンソルノイズ除去を実現する既存手法のひとつとして,t-SVD\cite{Kilmer}を用いる手法が挙げられる.この手法が用いるt-SVDとは,主に3階テンソルを入力とし,入力テンソルの第三次元に対して離散フーリエ変換(DFT)を行なったのち,テンソルを行列へと分解し,それらの行列を特異値分解することでテンソルの特異値分解を実現するものである.このとき,得られた特異値に対してしきい値処理を行うことでテンソルデノイズが実現できることが知られており,デノイズだけでなく一般のテンソル復元にも応用することができる\cite{Lu}.
この手法をさらに発展させ,従来用いられているDFTを,離散コサイン変換(DCT)および主成分分析(PCA)に置き換えることでテンソル復元性能を高めることが提案されている\cite{DCT, kohama}.一方で,一般のテンソル復元の一種である低ランクテンソル復元を対象とした先行研究\cite{kohama}では,復元対象のテンソルによって最適な直交変換が異なることが明らかとなっている.
\section{目的}
上記の背景より,本研究では復元対象のテンソルに適したノイズ除去結果を得ることを目的とし,DFT, DCT,PCAのなかから最良の直交変換を選択することで,これを実現する手法を提案する.提案手法では,前述の3つの直交変換のうち,最小二乗誤差の意味で原テンソルに最も近いテンソルを得られるものを最良の変換と定義する.しかしながら,実際にテンソルノイズ除去を行う際には原テンソルを参照することは不可能である.そこで本研究では,推定した行列と真の行列の平均二乗誤差を不偏推定するStein's unbiased risk estimator (SURE)を用いることでこの問題を解決する.